\section{Capítulo I: Descripción del proyecto}

Este capítulo nos presenta nuestro planteamiento del problema, el fundamento y los objetivos del estudio.
En primer lugar, presenta el panorama clínico y social de esta condición médica, enfocándose en la importancia epidemiológica y económica del fenómeno de las úlceras por presión en personas con movilidad reducida.
Y después, describe la relevancia de un aparato ajustable para la redistribución de la presión como prevención de tales lesiones.
Finalmente, se enuncian los objetivos del proyecto y los objetivos específicos para orientar el proyecto.


\subsection{Planteamiento del Problema de Investigación}

Las personas con movilidad limitada que se sientan durante períodos prolongados de tiempo tienen un mayor riesgo de desarrollar úlceras por presión (a veces llamadas úlceras de decúbito o escaras). También conocidas como úlceras de decúbito, estas lesiones ocurren cuando la presión sostenida limita el flujo de sangre a partes del cuerpo, causando necrosis del tejido.

Las úlceras por presión son un problema de salud importante en Colombia, ya que no solo reducen la calidad de vida de los pacientes y sus familias, sino que también contribuyen a los costos sociales y a los recursos del sistema de salud. Por lo tanto, también son un problema de salud y conllevan implicaciones legales significativas, ya que no pueden considerarse exclusivamente como un producto del cuidado de enfermería (\cite{minsalud2021}).

Las úlceras por presión no son solo un problema médico relacionado con el dolor y la calidad de vida de los pacientes, sino también un problema social de salud pública con implicaciones a nivel mundial debido a sus costos, su influencia en las familias y los cuidadores, la carga de cuidado pesada para las enfermeras, problemas legales que incluyen mediación y mortalidad. De hecho, la presencia de estas ulceras por presión conduce a un aumento de 2 a 4 veces en el riesgo de mortalidad per se (\cite{lorente2020}).

Los pacientes que se sientan en sillas están en alto riesgo debido a la posibilidad de que la presión y/o la fricción se ejerza sobre prominencias óseas. Se ha establecido en la investigación que esta dirección es responsable del 30\% de la prevalencia de las úlceras por presión en los pacientes hospitalizados (\cite{gefen2020}). Como se muestra en la Figura 1, la úlcera por presión existe en un 61\% bilateral en la región pélvica y un 38\% en la región sacra. Estos resultados son consistentes con informes previos en la literatura que han vinculado las regiones más afectadas por la movilidad reducida y la posición sentada en los pacientes (\cite{adillo2021}).


\begin{figure}[htbp]
    \centering
    \caption{Distribución de úlceras por presión según región anatómica.}
    \includegraphics[width=0.45\linewidth]{Distribución de úlceras por presión según región anatómica..png}
    \label{fig:distribucion_ulceras}\\
    % Si necesitas una nota para la figura (fuente, aclaraciones):
    {Fuente: (Adillo \& Parrilla, 2021). Nota: Este gráfico muestra la distribución porcentual de úlceras por presión por área anatómica.}
\end{figure}


Además, la incidencia de estas úlceras varía ampliamente según el entorno y la calidad de los cuidados preventivos. Por ejemplo, en entornos hospitalarios, la prevalencia de úlceras por presión puede oscilar entre un 4.5\% y un 28\%, dependiendo de la unidad de atención y la efectividad de las medidas preventivas implementadas (\cite{vangilder2017}). En hogares de ancianos, la prevalencia puede ser aún mayor, afectando a más del 23\% de los residentes (\cite{lyder2008}).\\

Aquellos estudios recientes muestran que no manejar bien estas úlceras no solo hace que los pacientes sufran más, sino que también aumenta mucho los costos de la atención médica. Según un informe de NHS England, entre abril de 2015 y marzo de 2016, se reportaron 24,674 nuevos casos de úlceras por presión en pacientes. Además, el tratamiento de estas lesiones cuesta al NHS más de £3.8 millones diarios (\cite{nhs2016}). Este dato resalta la importancia de implementar medidas preventivas efectivas para reducir tanto la incidencia de estas lesiones como los costos asociados con su tratamiento.\\

Adicionalmente, uno de los mayores desafíos en la prevención de úlceras por presión es la falta de cambios de posición regulares. Aunque se recomienda reposicionar a los pacientes cada 2 horas para aliviar la presión, un estudio reveló que solo el 40\% de los pacientes reciben estos cuidados de manera consistente (\cite{woodbury2004}). Esta deficiencia en la atención se debe, en parte, a la carga de trabajo del personal de enfermería y a la falta de sistemas de alerta eficaces (\cite{woodbury2004}). A pesar de las medidas preventivas actuales, la incidencia de escaras sigue siendo preocupantemente alta. Lo que contribuye significativamente a la prevalencia de estas lesiones y demuestra las limitaciones de las estrategias preventivas existentes. 

De esta manera, la prevención de las úlceras por presión es fundamental para mejorar la calidad de vida de los pacientes y reducir los costos asociados a su tratamiento. De hecho, estudios recientes han demostrado que hasta el 95\% de estas lesiones son evitables (\cite{sefh2023}). Sin embargo, a pesar de este conocimiento, la incidencia de úlceras por presión sigue siendo alta, lo que resalta la necesidad de implementar nuevas estrategias preventivas.\\

Por consiguiente, la necesidad de intervenciones efectivas, especialmente en el desarrollo de dispositivos que puedan monitorizar y redistribuir la presión de manera automática, es urgente para mejorar los resultados clínicos y reducir los costos asociados. Este proyecto tiene como objetivo diseñar y analizar un dispositivo que prevenga las úlceras por presión en pacientes con movilidad reducida en posición sedente, contribuyendo al cuidado y bienestar de estos pacientes. 


\begin{figure}[htbp]
    \centering
    \caption{Árbol de problemas: Riesgo de úlceras en pacientes con movilidad reducida.}
    \includegraphics[width=0.9\linewidth]{Árbol de problemas Riesgo de úlceras en pacientes con movilidad reducida..png}
    \label{fig:enter-label}\\
    {Fuente: Elaboración propia. }
\end{figure}

\subsubsection{Formulación o enunciado del problema.}
La revisión del contexto clínico, epidemiológico y tecnológico relacionado con las úlceras por presión en pacientes con movilidad reducida ha evidenciado la persistencia de esta problemática  a pesar de las estrategias preventivas actualmente implementadas. La alta prevalencia de estas lesiones  especialmente en la región glútea de pacientes en posición sedente, así como las limitaciones de los métodos tradicionales de prevención  justifican la necesidad de explorar soluciones innovadoras desde la ingeniería biomédica.\\
En este sentido  el desarrollo de dispositivos adaptativos que permitan redistribuir la presión de manera dinámica y personalizada, validado mediante simulaciones computacionales  se plantea como una alternativa prometedora. A partir de este análisis  surge la siguiente pregunta de investigación que orienta el presente trabajo:\\

 \textit{¿Cómo puede un dispositivo adaptativo, diseñado y validado mediante simulaciones computacionales, redistribuir la presión en la región glútea para prevenir escaras en pacientes con movilidad reducida en posición sedente?}
 
\subsubsection{Delimitación o alcance del problema}
Esta investigación se delimita conceptualmente al diseño virtual de un dispositivo biomédico adaptativo orientado a la redistribución de presión en la región glútea  con el fin de prevenir la formación de úlceras por presión en pacientes con movilidad reducida en posición sedente. El estudio se enfoca exclusivamente en la fase de simulación computacional del dispositivo, sin incluir pruebas clínicas ni validación en sujetos humanos.  No se abordan otras formas de intervención terapéutica ni dispositivos de tratamiento  sino únicamente estrategias preventivas basadas en redistribución de presión.\\

En cuanto a la delimitación temporal, el proyecto se desarrolla durante el año 2025  abarcando las fases de revisión bibliográfica, diseño conceptual, simulación y validación virtual del dispositivo. Las simulaciones se realizan en entornos computacionales controlados  sin intervención longitudinal ni seguimiento clínico posterior.\\

Espacialmente, la investigación se circunscribe a un entorno académico y de laboratorio  específicamente en la Universidad Manuela Beltrán, sede Bogotá  dentro del programa de Ingeniería Biomédica. No se contempla la implementación del dispositivo en instituciones de salud ni su aplicación directa en pacientes durante esta etapa.\\

Desde el punto de vista disciplinario, el estudio se enmarca en el campo de la Ingeniería Biomédica  integrando conocimientos de biomecánica  diseño asistido por computador (CAD), simulación por elementos finitos, electrónica aplicada y control automático. Asimismo  se apoya en fundamentos de la medicina preventiva y la fisiopatología de las úlceras por presión  sin pretender sustituir el juicio clínico ni las prácticas médicas establecidas.


\subsection{Justificación en términos de necesidades y pertinencia de la investigación}

Las úlceras por presión representan un problema de salud significativo que afecta severamente la calidad de vida de los pacientes con movilidad reducida, especialmente aquellos que permanecen en posición sedente por largos periodos. Según un estudio de (\cite{lyder2008}), la prevalencia de estas lesiones en entornos como los hogares de ancianos puede superar el 23\%, mientras que hasta un 95\% de los pacientes hospitalizados con movilidad reducida están en riesgo de desarrollarlas, lo que incrementa hasta cuatro veces la probabilidad de mortalidad (\cite{lorente2020}). Estas lesiones no solo causan dolor y sufrimiento físico, sino que también pueden derivar en complicaciones graves como infecciones, depresión y aislamiento social.

Además del impacto clínico, estas lesiones representan un desafío económico considerable. (\cite{padula2019}) estiman que el tratamiento de una úlcera por presión puede costar entre \$20,900 y \$151,700, generando una carga financiera significativa tanto para los sistemas de salud como para las familias de los pacientes, que frecuentemente deben asumir gastos adicionales de cuidado y tratamiento.

Para abordar esta problemática crítica, se propone el diseño de un dispositivo adaptativo destinado a monitorizar y redistribuir la presión en pacientes con movilidad reducida que permanecen en posición sedente. Este dispositivo integrará sensores para realizar un monitoreo continuo y ajustes dinámicos en tiempo real, previniendo así una de las principales causas de las lesiones por presión. La tecnología desarrollada no solo busca mejorar la calidad de vida de los pacientes y mitigar complicaciones, sino también optimizar el uso de recursos en entornos hospitalarios y de atención a largo plazo.

El proyecto se fundamenta en evidencias que subrayan la necesidad urgente de innovar en herramientas médicas preventivas (\cite{gefen2020}; \cite{liu2024}). La investigación se llevará a cabo durante un periodo de un año, abarcando las etapas de diseño conceptual, prototipado y validación en un entorno controlado. Estas actividades se realizarán en laboratorios de simulación médica y entornos de atención primaria en Colombia, cumpliendo con normativas locales e internacionales como el Decreto 4725 de 2005 y la ISO 13485:2016 (\cite{minproteccion2005}, \cite{iso13485_2016}).

La propuesta busca reducir la prevalencia de úlceras por presión, una condición que actualmente afecta a gran parte de la población con movilidad reducida y que, además, tiene un impacto significativo en los costos del sistema de salud. Al desarrollar una solución técnica basada en el monitoreo continuo y la redistribución dinámica de presión, se pretende no solo prevenir estas lesiones, sino también mejorar la eficiencia y sostenibilidad del cuidado de la salud.


\subsection{Objetivos: General y específicos}



\subsubsection{Objetivo general}
Diseñar virtualmente un dispositivo adaptativo para la redistribución de presión en la región glútea, orientado a prevenir la formación de escaras en pacientes con movilidad reducida en posición sedente.


\subsubsection{Objetivos específicos}
\begin{itemize}
    \item Identificar, a partir de revisión bibliográfica y análisis biomecánico, las zonas de mayor riesgo de presión en la región glútea de pacientes en posición sedente.


\item  Diseñar un sistema adaptativo de redistribución de presión que integre sensores y actuadores, tomando como base los parámetros biomecánicos identificados.


\item  Simular y validar el comportamiento del dispositivo diseñado mediante herramientas computacionales, evaluando su capacidad de redistribuir eficazmente la presión en las zonas. 


\end{itemize}


\vspace{2\baselineskip}