\section{Capitulo III: Marco metodológico}

Este capítulo describe el enfoque metodológico del proyecto, detallando las técnicas empleadas para la recolección y análisis de datos. También se exponen los procedimientos realizados en entornos de simulación biomecánica y electrónica, con el fin de evaluar virtualmente el comportamiento del dispositivo diseñado. Asimismo, se incluyen los métodos estadísticos aplicados a los resultados numéricos y las estrategias para garantizar la confiabilidad de los datos obtenidos. Todo esto permitió establecer una ruta técnica y coherente para valorar el impacto del dispositivo en la redistribución de presión en la región glútea de personas en posición sedente, como medida preventiva a la formación de escaras.

\subsection{Tipo de Investigación}
Esta investigación se clasifica como aplicada y tecnológica con un alcance \textbf{explicativo-descriptivo}. Es aplicada porque busca resolver un problema específico relacionado con la prevención de escaras en pacientes con movilidad reducida mediante el desarrollo de una solución tecnológica concreta. Es tecnológica porque se centra en el diseño, desarrollo y validación virtual de un dispositivo médico adaptativo.\\
El alcance explicativo se fundamenta en que el proyecto tiene como objetivo explicar cómo el diseño de un dispositivo puede prevenir la aparición de escaras mediante la redistribución de presión en zonas específicas del glúteo en posición sedente. Se busca identificar y explicar los efectos de la redistribución de presión en la prevención de escaras y, por tanto, se pretende ofrecer una explicación del fenómeno basado en la simulación del dispositivo.\\
Este alcance también incorpora elementos exploratorios y descriptivos. Es exploratorio porque se adentra en un área relativamente nueva buscando entender el comportamiento del dispositivo en diferentes condiciones simuladas. Es descriptivo porque busca detallar el funcionamiento del dispositivo, su capacidad para redistribuir la presión y su impacto en variables simuladas, como la presión en diferentes puntos del cuerpo.\\
La investigación se enmarca también dentro del paradigma \textbf{cuantitativo}, utilizando métodos de simulación computacional para generar datos numéricos medibles y comparables que permitan evaluar la eficacia del dispositivo propuesto.


\subsection{Diseño de Investigación}
El diseño de esta investigación es cuasi-experimental de simulación y transversal, específicamente un diseño de grupo único con mediciones pre-prueba y post-prueba. Tal diseño se justifica por las siguientes razones:
\begin{itemize}
    \item \textbf{Cuasi-experimental:} No se realizarán pruebas directas en personas, sino que se simulará el funcionamiento del dispositivo. Se planea manipular la variable independiente clave: la presencia o ausencia del dispositivo adaptativo. Esta manipulación permitirá establecer una relación causal entre el dispositivo y la redistribución de presión.
 \item \textbf{Transversal:} Se realizará la simulación del dispositivo en momentos dados sin intervención longitudinal (a lo largo del tiempo).
\item \textbf{Grupo único:} Se utilizará solamente un modelo de dispositivo para realizar las pruebas comparativas.
 \item \textbf{Pre-prueba y post-prueba:} Se medirá la presión en diferentes puntos antes y después de aplicar el dispositivo para evaluar el cambio presntado. De esta manera se pretende comparar la condición del sistema (la distribución de presión) antes y después de la intervención (aplicación del dispositivo), lo que permite evaluar el cambio producido por el mismo.
\end{itemize}

Adicionalmente, el subtipo del diseño es \textbf{descriptivo-explicativo}, ya que describe las propiedades del dispositivo y explica el efecto de la redistribución de presión en la prevención de escaras.\\
De la misma manera, la investigación incorpora un componente documental significativo, basado en revisión bibliográfica sistemática para fundamentar el diseño y validar los parámetros utilizados en las simulaciones.\\

\subsubsection{Fases de la Investigación}

El desarrollo de esta investigación se estructura en cuatro fases secuenciales e interconectadas, cada una con objetivos específicos y actividades claramente definidas a continuación. Esta organización metodológica permite un abordaje sistemático del problema, garantizando que cada etapa aporte elementos fundamentales para el logro del objetivo general del proyecto.\\
La secuencia de fases responde a una lógica de construcción progresiva del conocimiento, donde cada una proporciona los insumos necesarios para el desarrollo de la siguiente, culminando en un diseño validado virtualmente y documentado de manera integral.

\begin{table}[H]
\centering
\footnotesize
\caption{Fases del proyecto y sus actividades}
\label{tab:fases-proyecto}
\resizebox{\textwidth}{!}{%
\begin{tabular}{|c|m{3.5cm}|m{10cm}|}
\hline
\textbf{Fase} & \textbf{Nombre} & \textbf{Actividades} \\ \hline
1 & Recopilación de datos &
\begin{itemize}
    \item Revisión bibliográfica exhaustiva
    \item Estudio de zonas de riesgo en posición sedente
    \item Análisis de parámetros biomecánicos críticos
    \item Estudio de normativas aplicables
    \item Análisis de materiales compatibles
\end{itemize} \\ \hline
2 & Diseño Conceptual &
\begin{itemize}
    \item Definición de especificaciones técnicas
    \item Selección de sensores y actuadores
    \item Diseño preliminar del sistema electrónico
    \item Selección de materiales para cada componente
    \item Bocetos y modelado 3D inicial
\end{itemize} \\ \hline
3 & Simulación y Validación Virtual &
\begin{itemize}
    \item Simulación mecánica en COMSOL
    \item Simulación electrónica en Proteus
    \item Desarrollo y optimización de algoritmos de control
    \item Validación de parámetros de redistribución de presión
    \item Ajustes iterativos del diseño basados en simulación
    \item Análisis comparativo con dispositivos similares encontrados en buscadores académicos
    \item Validación con modelos antropométricos 
\end{itemize} \\ \hline
4 & Conclusiones y discusión &
\begin{itemize}
    \item Documentación técnica completa
    \item Informe final con conclusiones y recomendaciones
    \item Propuestas para futuras investigaciones
\end{itemize} \\ \hline
\end{tabular}%
}
\caption*{\textup{Fuente: Elaboración propia.\\
Nota: La tabla presenta la planificación general del proyecto de tesis por fases.}}
\end{table}
\textbf{Fase 1: Recopilación de datos}\\
Esta fase inicial constituye el fundamento teórico y técnico de la investigación. La revisión bibliográfica exhaustiva permite identificar el estado del arte en dispositivos para prevención de escaras, mientras que el estudio de zonas de riesgo en posición sedente proporciona la base anatómica necesaria para el diseño. El análisis de parámetros biomecánicos críticos establece los criterios cuantitativos que debe cumplir el dispositivo, y el estudio de normativas garantiza que el diseño cumpla con los estándares técnicos y de seguridad requeridos. Esta fase se concluye con el análisis de materiales compatibles encontrados en la literatura que serán utilizados en las simulaciones posteriores.\\
\textbf{Fase 2: Diseño Conceptual}\\
Con base en la información recopilada en la fase 1, se procede a la conceptualización técnica del dispositivo. La definición de especificaciones técnicas traduce los requerimientos biomecánicos en parámetros de ingeniería medibles. Luego,  selección de sensores y actuadores se fundamenta en los criterios de precisión, confiabilidad y compatibilidad identificados. El diseño preliminar del sistema electrónico integra todos los componentes en una arquitectura funcional, mientras que la selección final de materiales considera aspectos de biocompatibilidad, durabilidad y propiedades mecánicas, entre otros parámetros relevantes. La fase se culmina con el modelado 3D  inicial (CAD) que sirve como base para las simulaciones.\\
\textbf{Fase 3: Simulación y Validación Virtual}\\
Esta tecera fase representa el "núcleo" técnico de la investigación, donde se materializa virtualmente el diseño propuesto. La simulación mecánica en COMSOL permite evaluar el comportamiento biomecánico del dispositivo bajo diferentes condiciones de carga y uso. Paralelamente, la simulación electrónica en Proteus valida el funcionamiento del sistema de control y adquisición de datos. Tmabién, el desarrollo de algoritmos de control optimiza la respuesta del dispositivo ante cambios en la distribución de presión. Posteriormente la validación de parámetros de redistribución de presión constituye la evaluación central de la efectividad del diseño. Al conseguir lo anterior se pretende realizar los ajustes iterativos que permitan perfeccionar el diseño basándose en los resultados obtenidos en las simulaciones. La comparación con dispositivos similares encontrados en buscadores académicos proporcionará un marco de referencia para evaluar las ventajas del diseño propuesto, y la validación con modelos antropométricos garantizará la aplicabilidad clínica.\\
\textbf{Fase 4: Conclusiones y discusión}\\
La la última fase integra todos los resultados obtenidos en un análisis comprensivo. La documentación técnica completa sistematizará todos los aspectos del diseño, las simulaciones realizadas y los resultados obtenidos. También se pretende redactar un informe final que presentede manera estructurada las conclusiones sobre la efectividad del dispositivo y su potencial impacto en la prevención de escaras. Finalmente se darán propuestas para futuras investigaciones que identifiquen líneas de desarrollo que podrían continuarse, incluyendo la implementación física del prototipo y estudios clínicos.\\
\textbf{Interconexión entre fases}\\
Como se mencionó anteriormente, cada una de las fases se construye sobre los resultados de la anterior, creando un flujo de trabajo coherente y sistemático. Los datos recopilados en la Fase 1 informan las decisiones de diseño en la Fase 2. Los modelos conceptuales de la Fase 2 se implementan virtualmente en la Fase 3. Los resultados de simulación de la Fase 3 se analizan y documentan en la Fase 4. Esta estructura garantiza trazabilidad en las decisiones metodológicas y permite una validación progresiva del diseño propuesto.

Con el fin de visualizar de manera integrada el proceso metodológico propuesto para el desarrollo del dispositivo adaptativo, a continuación, se presenta el esquema metodológico general del proyecto, que sintetiza visualmente las etapas descritas anteriormente. Este pipeline resume la secuencia lógica y estructurada de las fases abordadas, desde la conceptualización hasta la validación de resultados, integrando los aspectos clave del proceso de investigación, diseño, simulación y evaluación.




\begin{figure}[h]
    \centering
    \caption{\textit{Pipeline fases del proyecto}}
    \includegraphics[width=1\textwidth]{Pipeline.png} % Reemplaza "ejemplo.png" con el nombre de tu archivo de imagen
    \\
    Fuente: Elaboración propia\\
    \label{fig:palma_aceite}
\end{figure}



\subsection{Población}
\subsubsection{ Población de Simulaciones}
La "población" de este estudio está constituida por todas las simulaciones posibles que pueden realizarse para evaluar la redistribución de presión en pacientes con movilidad reducida en posición sedente. En lugar de trabajar con sujetos humanos, este estudio se basa en modelos computacionales que permiten simular una variedad de escenarios.\\
En investigaciones similares encontradas en la literatura, se recomienda realizar múltiples simulaciones para garantizar la precisión del modelo predictivo reducir el sobreajuste y mejorar la confiabilidad de los resultados. La cantidad exacta de simulaciones depende de factores como la complejidad del modelo, la variabilidad de los datos y la precisión deseada. \\
%Estudios previos han mostrado que realizar al menos 100 simulaciones es una práctica común para obtener resultados robustos y confiables.\\

En este proyecto se propone realizar %aproximadamente \textbf{100 simulaciones}, 
 simulaciones distribuidas entre distintos escenarios predefinidos y sus repeticiones. Estas abarcarán:
\begin{itemize}
    \item \textbf{Diversidad de Escenarios:} Variación en los pesos corporales, tamaños anatómicos y materiales del dispositivo para reflejar una población representativa.
 \item \textbf{Repeticiones:} Cada escenario será simulado varias veces para garantizar la reproducibilidad y minimizar errores aleatorios.
 \item \textbf{Diferentes Condiciones del Sistema:} Incluyendo interacciones con superficies rígidas y blandas para evaluar cómo se comporta el dispositivo en diferentes entornos.
\end{itemize}
 
Consecuentemente, en este proyecto no habrá una muestra ni un proceso de muestreo tradicional, ya que como se mencionó previamente, no se probará el dispositivo en personas sino que se trabajará con un modelo simulado que puede repetirse tantas veces como sea necesario para obtener resultados confiables.


\subsubsection{Criterios de búsqueda de la literatura}
La población documental está compuesta por artículos científicos recientes, normativas nacionales e internacionales (como la Resolución 8430 de 1993 y la NTC ISO 13485), tesis académicas, informes técnicos y recursos multimedia sobre prevención de úlceras por presión, biomecánica en sedestación, diseño de dispositivos médicos y simulación computacional. Estas fuentes han sido consultadas en bases de datos como Scopus, Web of Science, ScienceDirect, IEEE Xplore y PubMed, garantizando calidad y actualidad.\\
\textbf{Criterios de inclusión documental:}
\begin{itemize}
    \item Publicaciones entre 2018–2025
\item Artículos en inglés en su mayoría (alrededor del 90\%) con pocas excepciones de artículos académicos específicos
\item Fuentes indexadas en bases científicas reconocidas (Scopus, Google Académico, Scielo, PubMed, Web of Science, Elsevier, ScienceDirect)
\item Contenidos relacionados con biomecánica, escaras, dispositivos adaptativos, COMSOL, Proteus o presión en sedestación
\end{itemize}

\textbf{Criterios de exclusión documental:}
\begin{itemize}
    \item Publicaciones anteriores a 2018 sin relevancia técnica o clínica
\item Documentos sin respaldo científico o de divulgación general
\item Fuentes que no aporten al objetivo preventivo del proyecto
\item Información no académica o sin respaldo científico
\end{itemize}

\subsubsection{Justificación de la Ausencia de Muestra Humana}
La ausencia de una muestra humana en este proyecto se justifica por los siguientes motivos:
\begin{enumerate}
    \item \textbf{Simulación Controlada:} El estudio utiliza un modelo simulado, lo que permite repetir la simulación tantas veces como sea necesario para obtener datos confiables y precisos, sin la variabilidad inherente a estudios con sujetos humanos.
\item \textbf{Control sobre Variables:} En un entorno simulado se tiene control total sobre las variables, lo que permite aislar el efecto del dispositivo y analizar su comportamiento en condiciones específicas.
\item \textbf{Objetivo de Evaluación del Dispositivo:} Se pretende evaluar el funcionamiento técnico del dispositivo en condiciones controladas. No se busca generalizar los resultados a una población humana en esta fase inicial, sino verificar el diseño del dispositivo y su impacto en la redistribución de presión para la prevención de escaras.
\item \textbf{Aspectos Éticos:} De acuerdo con la Resolución 8430 de 1993, este estudio se clasifica como investigación sin riesgo, ya que no involucra sujetos humanos ni manipulación de información confidencial.
\end{enumerate}



\subsection{Técnicas e instrumentos de recolección de datos}
\subsubsection{Instrumentos de Simulación}
\begin{enumerate}
    \item \textbf{ Software de simulación biomecánica (COMSOL Multiphysics):}
    \begin{itemize}
        \item Función: Simular la distribución de presión en el glúteo en diferentes escenarios y condiciones. Permite generar mapas de presión en función del diseño del dispositivo adaptativo.
 \item Datos recolectados: Mapas de presión y puntos críticos donde se concentran las mayores fuerzas, tanto antes como después de la intervención con el dispositivo. Valores numéricos de presión en Pascales (Pa) y distribución espacial de fuerzas.
    \end{itemize}
 
 \item \textbf{Simulador de circuito electrónico (Proteus):}
 \begin{itemize}
    \item Función: Simular el control de presión y la respuesta del sistema basado en los datos obtenidos de los sensores. Incluye simulaciones de los actuadores que redistribuyen la presión.
\item Datos recolectados: Variaciones en la presión a través del ajuste del dispositivo, tiempos de respuesta del sistema, y comportamiento de sensores y actuadores simulados. 
 \end{itemize}

 \item \textbf{Sensores de presión (simulados):}
 \begin{itemize}
     \item Función: Monitorear la presión en tiempo real en las zonas de riesgo para las escaras dentro del entorno de simulación.
\item Datos recolectados: Medidas de presión en Pascales en áreas específicas del glúteo, registros temporales de variación de presión.
 \end{itemize}
\end{enumerate}

\subsubsection{Dataset}

El dataset utilizado en el estudio \textit{In vivo strain measurements in the human buttock during sitting using MR-based digital volume correlation} \cite{zappala2024} fue seleccionado por su aporte y valor científico en el análisis tridimensional de la deformación de tejidos blandos en la región glútea durante la posición sedente. La elección se fundamenta en su enfoque metodológico, que combina imágenes de resonancia magnética (MRI) con correlación de volumen digital (DVC), lo que permite cuantificar con alta resolución espacial los desplazamientos y deformaciones internas del músculo glúteo mayor y la grasa subcutánea en condiciones progresivas de carga. Este trabajo científico no solo supera los enfoques tradicionales que se limitan a medidas externas o bidimensionales sino que también abre la posibilidad de observar fenómenos antes solo descritos cualitativamente como el deslizamiento lateral del glúteo respecto al tubérculo isquiático \cite{zappala2024}.

Este dataset puede ser utilizado como base para múltiples aplicaciones. En investigación biomecánica, puede servir para validar o refinar modelos por elementos finitos (FEM) de predicción de lesiones por presión, aportando datos reales de deformación tisular que permitan ajustar propiedades de materiales (cuán blando es el músculo, cuán compresible es la grasa) y condiciones de contorno \cite{linderganz2008,oomens2003}. En ciencias aplicadas al cuidado clínico, puede emplearse en el desarrollo de sistemas inteligentes de detección temprana de daño tisular, entrenando algoritmos de aprendizaje automático con mapas de deformación real \cite{doridam2018}. También puede utilizarse para comparar cómo diferentes regiones anatómicas (medial/lateral, grasa/músculo, profunda/superficial) responden ante cargas, facilitando así el diseño personalizado de soportes o superficies de alivio de presión \cite{brienza2018}.

En este sentido, el dataset se relaciona directamente con nuestro proyecto, que consiste en una simulación de un dispositivo adaptativo para la prevención de úlceras por presión en posición sedente. Ya que nos proporciona datos sobre cómo se deforman los tejidos blandos del glúteos bajo carga, que es precisamente el contexto biomecánico en el que funcionará el dispositivo. Nos permite por un lado utilizar esa información como base de referencia para validar las simulaciones del comportamiento del tejido en contacto con el sistema de soporte y por otro lado establecer umbrales de deformación considerados de riesgo que pueden ser incorporados en la lógica adaptativa del dispositivo. Además, los mapas tridimensionales de deformación permiten identificar regiones anatómicas críticas como el área sub isquiática que deben ser monitoreadas y aliviadas mediante sensores y mecanismos que redistribuyan la presión. Así de esta manera este dataset fortalece tanto la precisión biomecánica como la funcionalidad preventiva del diseño simulado.

No obstante, el dataset presenta algunas limitaciones que deben ser consideradas para mejorar su aplicabilidad y generalización. En primer lugar, la población estudiada está compuesta exclusivamente por varones jóvenes y sanos, lo que reduce su valor predictivo para grupos clínicamente más vulnerables, como personas mayores, mujeres o pacientes con movilidad reducida. En segundo lugar, la postura analizada (semi-recumbente) no refleja la posición sentada habitual en usuarios de sillas de ruedas, donde las lesiones por presión son más frecuentes. Además, la resolución axial del escáner (10 mm) requirió deconvolución digital para mejorar la calidad de los datos. En este caso, los autores usaron un software llamado NiftyMIC para interpolar o reconstruir información “faltante” entre cortes de 10 mm, de modo que las imágenes fueran más continuas y precisas en la dirección axial. Las cargas aplicadas fueron de corta duración (alrededor de 11 minutos), sin considerar la exposición prolongada característica de entornos clínicos reales \cite{zappala2024,neumann2015}.

Cómo aporte para un futuro trabajo sería ampliar este dataset incluyendo una muestra más diversa (sexo, edad, condición física), incorporar posturas verticales reales, aumentar el tiempo de carga y mejorar la resolución volumétrica del escaneo. Así, se avanzaría hacia un modelo de evaluación más representativo y aplicable en contextos preventivos y terapéuticos para úlceras por presión profundas \cite{gefen2008,ceelen2008}.




\subsubsection{Instrumentos de Revisión Documental}
\begin{enumerate}
    \item \textbf{Bases de datos científicas:}
    \begin{itemize}
        \item Scopus, Web of Science, ScienceDirect, IEEE Xplore, PubMed, Google Académico, SciELO
\item Función: Búsqueda sistemática de literatura especializada
\item Datos recolectados: Artículos científicos, normativas técnicas, estudios biomecánicos 
    \end{itemize}

 \item \textbf{Gestores bibliográficos:}
 \begin{itemize}
     \item Mendeley/Zotero para organización y citación de fuentes
\item Función: Sistematización de referencias y control de calidad documental
 \end{itemize}
\end{enumerate}
\subsubsection{Técnicas de Recolección}
\begin{enumerate}
    \item \textbf{Simulación pre y post intervención:}
 \begin{itemize}
     \item Realizar simulaciones sin el dispositivo adaptativo (pre-prueba) para determinar las zonas de alta presión
 \item Repetir la simulación después de la intervención (post-prueba) con el dispositivo activo para observar los cambios en la distribución de presión
  \item Datos recolectados: Comparación de los valores de presión antes y después de la intervención
 \end{itemize}

\item \textbf{Análisis de mapas de presión:}
\begin{itemize}
   \item Evaluar los mapas de presión generados en el software para identificar las áreas críticas que pueden desarrollar escaras
\item Datos recolectados: Puntos de alta presión, variaciones en diferentes posiciones del paciente
\end{itemize}

\item \textbf{Monitoreo de la efectividad de la redistribución:}
\begin{itemize}
    \item Usar las salidas de los sensores simulados para medir cómo el dispositivo reduce la presión en áreas críticas
\item 	Datos recolectados: Gráficas y estadísticas de reducción de presión en cada área
\end{itemize}

\item \textbf{Revisión bibliográfica sistemática:}	
\begin{itemize}
    \item Búsqueda estructurada en bases de datos con palabras clave específicas
\item Análisis crítico de literatura especializada
\item Síntesis de información relevante para el diseño
\end{itemize}

\end{enumerate}



\vspace{0.5cm}



\subsection{Técnicas de procesamiento y análisis de los datos}
\subsubsection{Procesamiento de Datos Cuantitativos}
Los datos numéricos obtenidos de las simulaciones se procesarán mediante técnicas estadísticas descriptivas e inferenciales. Los valores de presión se agruparán en intervalos para construir distribuciones de frecuencia y se calcularán medidas de tendencia central (media, mediana, moda) y de dispersión (desviación estándar, rango, coeficiente de variación).\\
\textbf{Software estadístico:} Se utilizará software especializado como R, SPSS o Python para el análisis estadístico de los datos, incluyendo:
\begin{itemize}
    \item 	Análisis descriptivo de distribuciones de presión
\item Pruebas de normalidad para variables continuas
\item Análisis comparativo pre-post mediante pruebas t pareadas
\item Análisis de varianza (ANOVA) para comparar múltiples escenarios
\item Análisis de correlación entre variables biomecánicas
\end{itemize}

\subsubsection{Métricas de Evaluación}
Las siguientes métricas cuantitativas serán calculadas para evaluar la efectividad del dispositivo:
\begin{itemize}
    \item \textbf{Presión máxima en zonas críticas (mmHg/Pa):} Valor máximo registrado en áreas de alto riesgo
 \item \textbf{Tiempo de respuesta del sistema (segundos):} Tiempo que tarda el dispositivo en redistribuir la presión ante cambios posturales
 \item \textbf{Uniformidad de distribución de presión:} Medida através de la desviación estándar de los valores de presión en la superficie de contacto
 \item \textbf{Porcentaje de reducción de presión:} Comparación cuantitativa entre condiciones con y sin dispositivo
 \item \textbf{Área de contacto efectiva (cm²):} Superficie total donde se distribuye el peso corporal 
\end{itemize}

\subsubsection{Análisis de Mapas de Presión}
Los mapas de presión generados en COMSOL se analizarán mediante:\\
\textbf{Análisis espacial:} Identificación de zonas críticas mediante código de colores y gradientes de presión. %Las áreas con valores superiores a 32 mmHg serán categorizadas como de alto riesgo según criterios clínicos establecidos. ( está sin cita)
\\
\textbf{Análisis temporal:}Evaluación de cambios en la distribución de presión a lo largo del tiempo de simulación, identificando patrones de comportamiento del dispositivo.\\
\textbf{Análisis comparativo:} Contraste directo entre mapas pre-intervención y post-intervención para cuantificar la efectividad de la redistribución.

\subsubsection{Tabulación y Presentación de Resultados}
Los datos se organizarán en tablas que muestren de forma ordenada los resultados obtenidos por los diferentes instrumentos, permitiendo visualizar el fenómeno estudiado con mayor claridad. Las tablas incluirán:
\begin{itemize}
    \item Valores promedio, máximos y mínimos de presión por zona anatómica
\item Comparaciones estadísticas entre escenarios
\item Análisis de efectividad por tipo de usuario simulado
\item Correlaciones entre variables biomecánicas
\end{itemize}

\textbf{Representación gráfica:} Se utilizarán gráficos de barras, líneas de tendencia, diagramas de dispersión y mapas de calor para facilitar la interpretación de resultados. Los gráficos incluirán intervalos de confianza y análisis de significancia estadística cuando corresponda.

\subsubsection{Procesamiento de Información Cualitativa}
Para la información documental recopilada se utilizarán matrices de análisis, cuadros comparativos y síntesis conceptuales para organizar los datos y hacerlos comprensibles. Esta información cualitativa será sistematizada mediante:
\begin{itemize}
    \item \textbf{Matrices de análisis bibliográfico:} Organización de fuentes por categorías temáticas
\item \textbf{Cuadros comparativos:} Contraste entre diferentes enfoques y metodologías encontradas en la literatura
\item \textbf{Síntesis conceptual:} Integración de conocimientos previos para fundamentar el diseño propuesto
\end{itemize}

\subsection{ Validación y Confiabilidad}
La confiabilidad de los resultados se garantizará mediante:
\begin{itemize}
    \item \textbf{Validación cruzada:} Comparación entre resultados de COMSOL y Proteus para verificar coherencia entre modelos biomecánicos y electrónicos.
\item \textbf{Repetibilidad:} Ejecución múltiple de simulaciones bajo las mismas condiciones para verificar estabilidad de resultados.
\item  \textbf{Validación con literatura:} Contraste de resultados obtenidos con valores reportados en estudios similares para verificar coherencia científica.
\end{itemize}



\vspace{0.5cm}


\subsection{Consideraciones Éticas}
Esta investigación ha sido aprobada por el Comité de Ética de la Universidad Manuela Beltrán, clasificándose como investigación sin riesgo según la Resolución 8430 de 1993, dado que no involucra experimentación con seres humanos ni manipulación de información confidencial. Se respetan los principios éticos de respeto, justicia, beneficencia y no maleficencia, así como las directrices institucionales y la Declaración de Helsinki.

\vspace{0.5cm}

\subsection{Modelo Anatómico}

Para el desarrollo del dispositivo adaptativo de redistribución de presion, se diseñó un modelo anatómico computacional que representa las estructuras críticas involucradas en la formación de úlceras por presión durante la sedestación prolongada. Este modelo se basa exclusivamente en parámetros anatómicos promedio y condiciones biomecánicas documentadas en la literatura científica, eliminando la necesidad de involucrar sujetos humanos en esta fase del estudio.

\subsubsection{Características del Modelo Anatómico}
El modelo se centra en las regiones anatómicas de mayor incidencia de úlceras por presión identificadas en estudios epidemiológicos: glúteos (tejido muscular y adiposo), sacro y tuberosidades isquiáticas (Li et al., 2014; Pradon et al., 2017). Los parámetros anatómicos utilizados se detallan en la siguiente tabla.

\begin{table}[H]
\centering
\caption{Parámetros anatómicos del modelo}
\label{tab:parametros-anatomicos}
\begin{tabular}{|p{4cm}|p{4cm}|p{5.5cm}|}
\hline
\textbf{Parámetro} & \textbf{Valor} & \textbf{Fuente bibliográfica} \\ \hline
Altura & 1.60 – 1.70 m & Li et al., 2013; Pradon et al., 2017 \\ \hline
Peso corporal & 60 – 70 kg & Li et al., 2013; Pradon et al., 2017 \\ \hline
Ancho de pelvis & 30 – 35 cm & Li et al., 2013; Pradon et al., 2017 \\ \hline
Área de contacto glútea & $\sim$40$\times$50 cm$^2$ & Pradon et al., 2017 \\ \hline
Tiempo crítico de exposición a presión & >2 horas & Chenu et al., 2013; Nair et al., 2020 \\ \hline
\end{tabular}

\vspace{1mm}
\centering
\caption*{\textup{Fuente: Elaboración propia con base en literatura científica.}}
\end{table}


Estos valores estandarizados fueron establecidos a partir de datos antropométricos publicados y respetan la segmentación corporal estándar descrita en la literatura biomecánica.

\subsubsection{ Desarrollo y Construcción del Modelo}
La construcción del modelo integra tres componentes metodológicos complementarios. Primero, el modelado computacional 3D mediante software CAD especializado para crear representaciones tridimensionales basadas en medidas antropométricas estándar. Segundo, la simulación por elementos finitos implementada en COMSOL Multiphysics para reproducir el comportamiento biomecánico de los tejidos, asignando propiedades viscoelásticas según valores publicados para piel, tejido adiposo, músculo y hueso. %Tercero, un prototipo físico construido con materiales que emulan las propiedades mecánicas de los tejidos humanos, utilizando siliconas de diferentes densidades para tejidos blandos y estructuras rígidas para componentes óseos.

\subsubsection{ Parámetros de Presión y Condiciones Simuladas}
Los datos de presión aplicados provienen exclusivamente de fuentes bibliográficas revisadas por pares. Se estableció una presión crítica para daño tisular de 70 mmHg (9.3 kPa) según Chenu et al. (2013) y Nair et al. (2020), con tiempos de exposición considerados riesgosos de 2 a 3 horas de presión continua. El modelo incorpora picos y gradientes de presión según patrones documentados en estudios de mapeo de presión en usuarios de sillas de ruedas, implementando comportamiento viscoelástico de tejidos mediante ecuaciones constitutivas basadas en investigaciones biomecánicas publicadas.
\subsubsection{Validación del Modelo}
La validación se realizó mediante comparación con datos publicados según los métodos descritos en la Tabla "Métodos de validación del modelo".

\begin{table}[H]
\centering
\caption{Métodos de validación del modelo}
\label{tab:validacion-modelo}
\begin{tabular}{|p{4.2cm}|p{5.3cm}|p{4cm}|}
\hline
\textbf{Método de Validación} & \textbf{Descripción} & \textbf{Fuente Bibliográfica} \\ \hline
Mapeo de presión experimental & Comparación con distribuciones documentadas en cojines anti-decúbito & Li et al., 2014; Pradon et al., 2017 \\ \hline
Simulaciones previas & Comparación con otros modelos de elementos finitos publicados & Li et al., 2013; Brunzini et al., 2023 \\ \hline
Dispositivos clínicos comerciales & Comparación con datos de sistemas como TexiCare y colchones anti-decúbito & Chenu et al., 2013; Nair et al., 2020 \\ \hline
\end{tabular}
\centering
\caption*{\textup{Fuente: Elaboración propia con base en literatura científica.}}
\end{table}

Esta aproximación metodológica garantiza que el modelo reproduzca adecuadamente las condiciones biomecánicas requeridas para el desarrollo del dispositivo de este proyecto.

\subsubsection{Limitaciones Reconocidas}
No obstante, el modelo presenta limitaciones inherentes que incluyen la simplificación anatómica mediante representación idealizada sin variaciones individuales, ausencia de variabilidad fisiológica como microcirculación tisular o estado nutricional, enfoque en condiciones estáticas sin micro-movimientos naturales, homogeneización de tejidos con propiedades simplificadas, y validación indirecta basada en comparación con datos publicados. Estas limitaciones son consideradas aceptables para los objetivos de esta investigación y coherentes con metodologías similares en el campo.
\subsubsection{Justificación Metodológica}
La utilización de este modelo anatómico se justifica por su capacidad para proporcionar resultados reproducibles en condiciones controladas, permitir la evaluación del dispositivo de manera segura antes de consideraciones futuras de validación clínica, y mantener coherencia con tendencias actuales en investigación biomédica que priorizan simulaciones computacionales en fases de desarrollo. Esta aproximación establece una base sólida para el diseño y optimización del dispositivo adaptativo de redistribución de presión, cumpliendo con los estándares éticos y metodológicos requeridos para investigaciones en el área biomédica.

