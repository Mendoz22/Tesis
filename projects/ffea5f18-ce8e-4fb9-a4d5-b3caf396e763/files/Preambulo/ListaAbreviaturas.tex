\section*{Lista de Abreviaturas}

\begin{adjustbox}{center}
\begin{tabular}{ll}
\toprule
\textbf{Abreviatura} & \textbf{Significado} \\
\midrule
AI & Inteligencia Artificial \\
CAD & Diseño Asistido por Computador \\
CE & Conformidad Europea \\
CNN & Convolutional Neural Network (Red Neuronal Convolucional) \\
COMSOL & COMSOL Multiphysics (software de simulación por elementos finitos) \\
FDA & Food and Drug Administration (Administración de Alimentos y Medicamentos) \\
IEEE & Institute of Electrical and Electronics Engineers \\
ISO & International Organization for Standardization \\
K-NN & K-Nearest Neighbors \\
kPa & Kilopascal (1 kPa = 1,000 Pa) \\
NHS & National Health Service (Servicio Nacional de Salud del Reino Unido) \\
NTC & Norma Técnica Colombiana \\
Pa & Pascal (unidad de presión) \\
R & Lenguaje de programación y software estadístico \\
Scielo & Scientific Electronic Library Online \\
SGSSS & Sistema General de Seguridad Social en Salud \\
SOGCS & Sistema Obligatorio de Garantía de Calidad en Salud \\
SPSS & Statistical Package for the Social Sciences (software estadístico) \\
SVM & Support Vector Machine \\
UPP & Úlceras por presión \\
\bottomrule
\end{tabular}
\end{adjustbox}




%Nota: Esta sección es opcional, dado que existen disciplinas que no manejan símbolos y/o abreviaturas. Se incluyen símbolos generales (con letras latinas y griegas), subíndices, superíndicesy abreviaturas (incluir sólo las clases de símbolos que se utilicen). Cada una de estas listas debe estar ubicada en orden alfabético de acuerdo con la primera letra del símbolo