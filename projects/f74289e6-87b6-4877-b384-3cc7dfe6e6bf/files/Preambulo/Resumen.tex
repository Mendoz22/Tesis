\section*{Resumen}
\vspace{0.5cm}
\noindent Las úlceras por presión, también conocidas como escaras, son reconocidas como lesiones crónicas de la piel que aparecen como consecuencia de una presión sostenida sobre determinadas zonas del cuerpo, siendo frecuentes en personas con movilidad reducida. Estas lesiones tienden a aparecer en la región glútea, sobre todo cuando el paciente permanece sentado durante largos periodos de tiempo. Este trabajo tiene como propósito desarrollar, de manera virtual, un dispositivo adaptativo orientado a redistribuir la presión en la zona glútea, con el fin de prevenir la aparición de escaras en pacientes que deben mantener la sedestación por tiempos prolongados. Entendiendo ya nuestro objetivo a realizar, se llevó a cabo una revisión bibliográfica sistemática que permitió identificar los principales factores de riesgo, donde se complementó con un análisis biomecánico de las áreas más propensas a ser vulnerables.Teniendo en cuenta esta base literaria, se planteó el diseño de un sistema compuesto por sensores de presión, actuadores neumáticos y un algoritmo de control inteligente. La validación del prototipo se realizó a través de simulaciones computacionales en plataformas como \textit{COMSOL} y \textit{Proteus}, utilizando modelos digitales con parámetros anatómicos y fisiológicos tomados de bases de datos científicas reconocidas. Es importante resaltar que no se hicieron pruebas en pacientes reales ni en muestras biológicas humanas.Las simulaciones se centraron en evaluar cómo se distribuye la presión en distintas posturas, así como la capacidad del sistema para responder a cambios en la posición del usuario y su respectiva efectividad en aliviar la presión en las zonas críticas. Se espera que los resultados obtenidos nos guíen a diseñar una base para el desarrollo de soluciones tecnológicas aplicables en el entorno clínico, ofreciendo una alternativa preventiva innovadora para el cuidado de personas en condición de sedestación prolongada.

\vspace{0.5cm}

\noindent Palabras clave: úlceras por presión, sedestación prolongada, redistribución de presión, simulación computacional, dispositivo biomédico.

\vspace{3cm}

\section*{Abstract}
\vspace{1cm}
\noindent Pressure ulcers, also known as bedsores, are recognized as chronic skin lesions that result from sustained pressure on specific areas of the body. They are particularly common among individuals with limited mobility. These lesions tend to appear most frequently in the gluteal region, especially when a patient remains seated for extended periods of time.This work aims to virtually develop an adaptive device designed to redistribute pressure in the gluteal area in order to prevent the onset of pressure ulcers in patients who must remain in a seated position for prolonged durations. With this objective in mind, a systematic literature review was conducted to identify the main risk factors, which was complemented by a biomechanical analysis of the most vulnerable areas.Based on this theoretical foundation, a system was designed that integrates pressure sensors, pneumatic actuators, and an intelligent control algorithm. The prototype was validated through computational simulations using platforms such as \textit{COMSOL} and \textit{Proteus}, employing digital models with anatomical and physiological parameters obtained from recognized scientific databases. It is important to note that no tests were conducted on real patients or biological specimens.The simulations focused on evaluating how pressure is distributed in various postures, the system’s ability to respond to changes in user position, and its effectiveness in relieving pressure in critical zones. The results are expected to provide a foundation for the development of technological solutions applicable in clinical settings, offering an innovative preventive approach for the care of individuals subjected to prolonged sitting.



\vspace{1cm}

\noindent Keywords:pressure ulcers, prolonged sitting, pressure redistribution, computational simulation, biomedical device.