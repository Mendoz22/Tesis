\section{Capítulo IV: Análisis y discusión de resultados}
Consiste en el estudio detallado y sistemático de los datos debidamente organizados en gráficos y tablas, con el objeto de encontrar las causas y efectos del fenómeno estudiado.

\subsection{Presentación de los resultados}
Puede ser a través de gráficos o tablas.

\subsection{Análisis e interpretación de los resultados}
Se resumen en la significación de cada uno de los gráficos y cuadros presentados. Una vez recopilada la información, debe hacerse de inmediato su procesamiento, esto alude a la forma de ordenar y presentar de la manera más lógica y clara, los resultados obtenidos con los instrumentos aplicados, logrando que la variable refleje el peso específico de su magnitud. El objetivo final es construir con dicha información: cuadros estadísticos, promedios generales y gráficos ilustrativos en los que se sinteticen los valores y pueda a partir de ellos, extraer enunciados teóricos. En conclusión, el procesamiento es el registro de los datos obtenidos, mediante una técnica analítica en la cual se comprueba la hipótesis y se obtienen conclusiones.

\subsection{Discusión}
La discusión es la etapa que encadena los resultados obtenidos por la investigación y la extrapolación de estos. En ella se pone a prueba la capacidad analítica y de autocrítica del autor y donde éste tiene la libertad de expresión. La discusión, cuando está bien formulada, extiende el ámbito de interés, hace posible que los lectores accedan al marco teórico y al conocimiento previo existente para la interpretación de los resultados, reconoce las limitaciones de la investigación o abre el camino a nuevas hipótesis o propuestas teóricas. La discusión pone el toque personal al trabajo e ilustra de manera clara, los resultados de la investigación frente a los teóricos que sustentan el proyecto o trabajo de investigación.

\subsection{Conclusiones y recomendaciones}
Es llamado también síntesis y es básicamente la interpretación final de los datos con los cuales se cierra la investigación iniciada. Las conclusiones deben plantearse con un alto margen de seguridad, por lo cual es recomendable usar términos afirmativos. Las recomendaciones son sugerencias que suelen hacerse para optimizar los logros obtenidos o mejorar aquello que no se consiguió en la investigación, estableciendo una relación directa con las conclusiones.
